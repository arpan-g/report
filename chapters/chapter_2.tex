\chapter{Methodology}
\label{chp:CHAPTERTITLE}
This chapter describes the methodology adapted in obtatinnig the sensor placement in the given grid. 
First section describes the data that is obtained from the PIR sensors.
Second section describes the feature that is extracetd from the data.
Third section gives detailed explanation of the algorithm.


\section{Sensor Data}
\label{sec:sensorData}
The pir sensor gives a bianry output of 1's and 0's.

Every caption of a table (or figure) should start with a capital letter, and should end with a period. References to tables are given with a capital letter for table, as in ``(see Table~\ref{tab:EXAMPLETABLE})'' or ``in Table~\ref{tab:EXAMPLETABLE}, ...''.

\begin{table}[htb]
\centering
\begin{tabular}{|l|c|r|}
\hline % horizontal line
left aligned & centred & right aligned \\
\hline \hline
12           & 34      & 56            \\
\hline
\end{tabular}
\caption{Complete sentence describing the tabular data.}
\label{tab:EXAMPLETABLE}
\end{table}

References to figures are given with a capital letter for figure, as in ``(see Figure~\ref{fig:EXAMPLEFIGURE})'' or ``in Figure~\ref{fig:EXAMPLEFIGURE}, ...''.

\cite{b}
\cite{a}

\begin{figure}[htb]
% most GNUplot figures need to be rotated, width should be the same throughout the complete document, and no extension is needed
\includegraphics[angle=180,width=\textwidth]{pics/TUD_logo_zw}
\caption{Complete sentence describing the figure thoroughly.}
\label{fig:EXAMPLEFIGURE}
\end{figure}

