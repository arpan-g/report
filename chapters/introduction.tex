\chapter{Introduction}
\label{chp:introduction}

The European Union (EU) has pledged to cut the consumption of primary energy by 20\% by the year 2020.  It is estimated that buildings consume 40\% of the energy produced\footnote{according to value published at https://ec.europa.eu/energy/en/topics/energy-efficiency/buildings }.  This has resulted in an increase in the demand to reduce the energy consumption of buildings. To reduce the consumption of energy, building automation systems (BAS) are being widely employed. BAS are computer-based systems that help to manage, control and monitor building technical services (HVAC, lighting etc.) and the energy consumption of devices used by the building.  BAS deploy huge amount of sensors, which provide inputs to perform efficient control of various services like HVAC and lighting. BAS brings with it various benefits,at the same time, offers numerous challenges too.One of the primary challenges is generating and updating the locations of the installed sensors.  For performing effective control, information about the location is highly important. As the size and distribution of the deployed sensors are high, it is highly cumbersome and error prone to manually maintain the meta-data about the sensor placement. Also as building evolve and change managing this spatial information becomes a cumbersome process. Hence to ease the process of sensor location verification an automatic process to locate the sensors in a building is required.


%\section{Application}
%FOr few of the applications like the one developed in 




\vspace{1\baselineskip}

\noindent
%TODO ORGANISATIONAL DESCRIPTION OF THESIS

