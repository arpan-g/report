\chapter{Literature Review}
%In \cite{wang2010survey}  \citeauthor{wang2010survey} classify the problem of sensor localization into three categories: proximity based localization , range based localization and angle based localization.\\
%In \textbf{proximity based localization} few nodes are aware of there locations called the anchor nodes and the other nodes calculate their position relative to these anchor nodes.\\
%\textbf{Range based localization:} This localization is  based on the ranging techniques using RSSI, Time of Arrival, Time Difference of Arrival . Although computation of the distances between each pair of location is trivial, the problem of finding the locations given the pairwise Euclidean distance is not. Range based approach may %or may not need anchor nodes. With anchor based approach Multilateration technique is used to find the location of the non anchor nodes. Without anchor nodes multi dimensional scaling (MDS) is adopted for localization. \\
%\textbf{Angle based localization:} In Angle based localization method additional information of angle of arrival of the signal is used for the process of localization. To caclulate the angle of arrival antenna array or multiople receivers on the nodes are required.\\ 
%Though the above discussed methods are known to give good results in localizing the sensor networks. Theses methods are not likely to be applied in a large indoor environment like commercial office spaces . As these methods either need specialized hardware in the case of angle based localization or they need higher processing %power. Which result in the increase of the cost per sensor node.This gives rise for a need of an alternate solution. The other method of localizing the sensors within a facility is sensor data driven.\\ 
%This section provides an insight about the work that has been done in automatic identification of the sensor locations within a building using sensor data. Previous efforts on creating a data-driven sensor localization methods have identified ways to cluster the sensors that are located within the same room.However the existing %methods do not identify the locations of the sensors within the building. Previous work of \citeauthor{Hong:2013:TAS:2528282.2528302} and \citeauthor{fontugne2012empirical} \cite{fontugne2012empirical} have shown methods of clustering different sensors that fall within a room.\\ 
%In particular \citeauthor{Hong:2013:TAS:2528282.2528302}\cite{Hong:2013:TAS:2528282.2528302} extended the work done in \cite{fontugne2012empirical} and showed the existence of a statistical boundary between sensors analogous to the physical one exists and is empirically discoverable.
%In \cite{doi:10.1061/9780784413616.226} \citeauthor{doi:10.1061/9780784413616.226} propose a feature, energy content in HVAC delivered air, which can be derived from HVAC system sensors which could lead to identification of the space in which the sensors are located . \citeauthor{Lu:2014:SBS:2648771.2629441}%%%\cite{Lu:2014:SBS:2648771.2629441} and \citeauthor{ellis2012creating}\cite{ellis2012creating} describe methods to obtain floor map for the smart home from the sensor data. They are able to map the sensors to the room and obtain connectivity map between the rooms in the smart home.\\
%All the previous work look at identifying the sensors that are present in same room but they do not answer the question of which sensor is located where within the given space. In our work we propose a method to determine the position of the sensors within an indoor space given the possible locations of the sensors.

There has been an extensive body of research that aims at obtaining the location information of the sensors in WSN's. All the approach that has been taken till now has been categorized by \citeauthor{wang2010survey} in \cite{wang2010survey} to fall into one of the following categories:
\begin{itemize}
\item Proximity based localization
\item Range based localization
\item Angle based localization
\end{itemize}

In proximity based localization, WSN is represented as a graph $G(V,E)$. Location of the  subset of the nodes, $H \subset G$ are known. The goal is to estimate the location of the remaining $V-H$ nodes relative to the position of the $H$ nodes. 
Range based localization makes use of  ranging techniques using RSSI, time of arrival and time difference of arrival of the signals. The distance is computed using the ranging technique. Computing the locations of the nodes using distance information is non trivial. Range based approach may or may not need anchor nodes. With anchor based approach Multilateration technique is used to find the location of the non anchor nodes. Without anchor nodes, multi dimensional scaling (MDS) is adopted for localization. 
Angle based  localization uses the information of angle of arrival of the signals to determine the location of the sensors. To determine the angle, antenna array or multiple receivers on the node are required.

Apart from the techniques mentioned above, recently there have been few works which look at the data measured by the sensors to obtain the information about the location of the sensor. As our approach towards determining the sensor location uses sensor data as input, in this chapter we provide a brief overview of the works which take the same approach. 

Various approaches have been taken to automate the process of determining the sensors location using the data from the sensors in buildings using different data analytics and signal processing tools. \citeauthor{Hong:2013:TAS:2528282.2528302}\cite{Hong:2013:TAS:2528282.2528302}
 apply empirical mode decomposition to 15 sensors in 5 rooms to cluster sensors which belong to the same room by analyzing the correlation coefficients of the intrinsic mode functions. They characterize the correlation coefficient distribution of sensors that are located in the
 same room and different rooms. They were able to show that there exists a correlation boundary analogous to the physical boundary which can be discovered empirically. 
In \cite{doi:10.1061/9780784413616.226} \citeauthor{doi:10.1061/9780784413616.226} propose a feature: energy content in HVAC delivered air, which can be derived from HVAC system sensors which could lead to the identification of the space in which the sensors are located . They combine sensor measurements and building characteristics(floor area) for the identification of the space in which the sensors are present.
In \cite{Koc:2014:CLC:2674061.2674075} \citeauthor{Koc:2014:CLC:2674061.2674075} propose a method to automatically identify the zone temperature and discharge temperature sensors that are closest to each other by using a statistical method on the collected raw data. They explore whether linear correlation or a statistical dependency measure(distance correlation) are better suited to infer spatial relationship between the sensors. They carry out their analysis on three different testbeds. They also investigate the effects of distance between sensors and measurement periods on the matching results. In the end, the authors conclude that linear correlation coefficient provides better matching results compared to distance correlation. The authors also conclude that as the distance between the sensors increase, the data size needed to infer spatial relationships also increase.

\citeauthor{Lu:2014:SBS:2648771.2629441}\cite{Lu:2014:SBS:2648771.2629441} describe a method to generate representative floor plans for a house. Their method clusters sensors to a room and assigns connectivity based on the simultaneous firing of the sensors placed on the door and window jambs. The algorithm gives a small set of possible maps from which the user has to choose the right map. Method requires special placement of the sensors . The authors were able to calculate the floor map of 3 out of the 4 houses they evaluated.  
 \citeauthor{ellis2012creating}\cite{ellis2012creating} proposed an algorithm to compute the room connectivity using  PIR and light sensor data. They compute room connectivity based on the artificial light spill over between rooms; occupancy detection due to movements between
  rooms; and fusion of the two. They calculate the transition matrix for the light sensor and occupancy sensor. Fuse both the data together to compute the connectivity graph. Here the authors have considered a situation where there is only one PIR and light sensor per room.
\citeauthor{muller2014automated} define  sensor topology as a graph with directed and weighted edges. All pairs of consecutive sensors triggers are interpreted as a user walking from the former to the latter sensing region indicated in the sensor 
graph by an edge from the former to the latter. Every time a consecutive edge triggers are observed the weight of the edge between the sensors is incremented. They define a method to filter out erroneous edges.
In \cite{marinakis2005learning} \citeauthor{marinakis2005learning} obtain the sensor network topology using Monte Carlo Expectation Maximization. They assign activity to people present in the space to obtain the graph topology. The algorithm requires, number of people present in the space as the input. They demonstrate the effectiveness of the algorithm using various simulated data. 




%\section{Subgraph Isomorphism}
%
%In our work we reduce the problem of mapping the sensors to its location in the grid to graph matching problem. A graph matching process between two graphs $G$ = $(V_G,E_G)$ and H = $(V_H,E_H)$ consists of determining a mapping M which associates nodes of the graph G to H and vice versa. Different constraints can be imposed onto M which results in different mapping types: monomorphism, isomorphism, graph-subgraph isomorphism are the most popular ones.Graph matching is widely used in pattern recognition for comparing the graph representing an object to a model  graph, or prototype. Our problem falls under the category of graph monomorphism .  
%Most of the algorithms for graph matching are based on some form of tree search with backtracking. The basic idea is that the partial mapping (initially empty) is iteratively expanded by adding to it new pairs of matched nodes; the pairs are chosen based on some conditions employed to satisfy the conditions of the matching type.
%The first significant and most widely used algorithm in the area of graph matching was proposed by \citeauthor{Ullmann:1976:ASI:321921.321925} \cite{Ullmann:1976:ASI:321921.321925}. Ullmann algorithm addresses the problem of graph isomorphism, sub graph isomorphism , graph monomorphism. To prune unfruitful matches the author proposes a refinement procedure, which employ conditions based on the knowledge of the adjacent nodes and the degree of the nodes that are being matched. 
%\citeauthor{4308468} in \cite{4308468} propose a graph monomorphism algorithm. The major drawback of the algorithm is that the algorithm uses a $N^2 \times N^2$ matrix to represent netgraph
%. N represent the number of nodes of the largest graph. As a result of this only small graphs can be dealt with using the algorithm.  
%A more recent algorithm for graph isomorphism, subgraph isomorphism and monomorphism was proposed by \citeauthor{906251} in \cite{906251}, popular as VF algorithm. The authors define a heuristic that is based on the analysis of sets of nodes adjacent to the ones already considered in the partial mapping. The authors further improved the algorithm in 2001 in there paper \cite{cordella2001improved}. 
%The authors proposed a modification to the algorithm(called VF2) which reduces the memory consumption from $O(N^2)$ to O(N) where N is the number of nodes in the graph, thus enabling the algorithm to work with large graphs.
%In our work we use the VF2 algorithm to obtain the mappings between the spanning tree and grid graph.