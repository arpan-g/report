\chapter{Literature Review}
%In \cite{wang2010survey}  \citeauthor{wang2010survey} classify the problem of sensor localization into three categories: proximity based localization , range based localization and angle based localization.\\
%In \textbf{proximity based localization} few nodes are aware of there locations called the anchor nodes and the other nodes calculate their position relative to these anchor nodes.\\
%\textbf{Range based localization:} This localization is  based on the ranging techniques using RSSI, Time of Arrival, Time Difference of Arrival . Although computation of the distances between each pair of location is trivial, the problem of finding the locations given the pairwise Euclidean distance is not. Range based approach may %or may not need anchor nodes. With anchor based approach Multilateration technique is used to find the location of the non anchor nodes. Without anchor nodes multi dimensional scaling (MDS) is adopted for localization. \\
%\textbf{Angle based localization:} In Angle based localization method additional information of angle of arrival of the signal is used for the process of localization. To caclulate the angle of arrival antenna array or multiople receivers on the nodes are required.\\ 
%Though the above discussed methods are known to give good results in localizing the sensor networks. Theses methods are not likely to be applied in a large indoor environment like commercial office spaces . As these methods either need specialized hardware in the case of angle based localization or they need higher processing %power. Which result in the increase of the cost per sensor node.This gives rise for a need of an alternate solution. The other method of localizing the sensors within a facility is sensor data driven.\\ 
%This section provides an insight about the work that has been done in automatic identification of the sensor locations within a building using sensor data. Previous efforts on creating a data-driven sensor localization methods have identified ways to cluster the sensors that are located within the same room.However the existing %methods do not identify the locations of the sensors within the building. Previous work of \citeauthor{Hong:2013:TAS:2528282.2528302} and \citeauthor{fontugne2012empirical} \cite{fontugne2012empirical} have shown methods of clustering different sensors that fall within a room.\\ 
%In particular \citeauthor{Hong:2013:TAS:2528282.2528302}\cite{Hong:2013:TAS:2528282.2528302} extended the work done in \cite{fontugne2012empirical} and showed the existence of a statistical boundary between sensors analogous to the physical one exists and is empirically discoverable.
%In \cite{doi:10.1061/9780784413616.226} \citeauthor{doi:10.1061/9780784413616.226} propose a feature, energy content in HVAC delivered air, which can be derived from HVAC system sensors which could lead to identification of the space in which the sensors are located . \citeauthor{Lu:2014:SBS:2648771.2629441}%%%\cite{Lu:2014:SBS:2648771.2629441} and \citeauthor{ellis2012creating}\cite{ellis2012creating} describe methods to obtain floor map for the smart home from the sensor data. They are able to map the sensors to the room and obtain connectivity map between the rooms in the smart home.\\
%All the previous work look at identifying the sensors that are present in same room but they do not answer the question of which sensor is located where within the given space. In our work we propose a method to determine the position of the sensors within an indoor space given the possible locations of the sensors.


\section{Sensors Localization in Buildings} 
Various approaches have been taken to automate the process of determining the sensors location in buildings using various data analytics and signal processing tools. \citeauthor{Hong:2013:TAS:2528282.2528302}\cite{Hong:2013:TAS:2528282.2528302} apply empirical mode decomposition to 15 sensors in 5 rooms to cluster sensors which belong to the same room by analyzing the correlation coefficients of the intrinsic mode functions. They characterize the correlation coefficient distribution of sensors in the same room and different rooms and show that there exists a correlation boundary analogous to the physical boundary and can be discovered empirically. \cite{doi:10.1061/9780784413616.226} \citeauthor{doi:10.1061/9780784413616.226} propose a feature: energy content in HVAC delivered air, which can be derived from HVAC system sensors which could lead to identification of the space in which the sensors are located . They combine sensor measurements and building characteristics(floor area) .\\ \citeauthor{Lu:2014:SBS:2648771.2629441}\cite{Lu:2014:SBS:2648771.2629441} describe a method to generate representative floor plans for a house. Their method clusters sensors to room and assigns connectivity based on simultaneous firing of the sensors placed on the door and windows jamb. The algorithm gives a small set of possible maps from which the user has to choose the right map. The authors were able to calculate the floor map of 3 out of the 4 houses they evaluated.  There method requires special placement of the sensors .
 \citeauthor{ellis2012creating}\cite{ellis2012creating} proposed an algorithm to compute the room connectivity using  PIR and light sensor data. They compute room connectivity based on the artificial light spill over between rooms ; occupancy detection due to movements between two rooms. They calculate the transition matrix for light sensor and occupancy sensor. Fuse both the data together to compute the connectivity graph. Here the authors have considered a situation where there is only one PIR and light sensor per room.


\section{Subgraph Isomorphism}
In our work we reduce the problem of mappings the sensors to its location in the grid to graph monomorphism problem. Graph monomorphism is widely used in pattern recognition for comparing the graph representing an object to a model  graph, or prototype. Various algorithms have been developed so far to solve the problem of graph monomorphism. In our work here we make use of the VF2\cite{cordella2001improved} algorithm to solve for monomorphism.